\documentclass[]{article}
\usepackage{lmodern}
\usepackage{amssymb,amsmath}
\usepackage{ifxetex,ifluatex}
\usepackage{fixltx2e} % provides \textsubscript
\ifnum 0\ifxetex 1\fi\ifluatex 1\fi=0 % if pdftex
  \usepackage[T1]{fontenc}
  \usepackage[utf8]{inputenc}
\else % if luatex or xelatex
  \ifxetex
    \usepackage{mathspec}
  \else
    \usepackage{fontspec}
  \fi
  \defaultfontfeatures{Ligatures=TeX,Scale=MatchLowercase}
\fi
% use upquote if available, for straight quotes in verbatim environments
\IfFileExists{upquote.sty}{\usepackage{upquote}}{}
% use microtype if available
\IfFileExists{microtype.sty}{%
\usepackage{microtype}
\UseMicrotypeSet[protrusion]{basicmath} % disable protrusion for tt fonts
}{}
\usepackage[margin=1in]{geometry}
\usepackage{hyperref}
\PassOptionsToPackage{usenames,dvipsnames}{color} % color is loaded by hyperref
\hypersetup{unicode=true,
            pdftitle={Rigdon Basu (1989) Time truncated data example},
            pdfauthor={Miao Cai miao.cai@slu.edu},
            colorlinks=true,
            linkcolor=blue,
            citecolor=Blue,
            urlcolor=Blue,
            breaklinks=true}
\urlstyle{same}  % don't use monospace font for urls
\usepackage{color}
\usepackage{fancyvrb}
\newcommand{\VerbBar}{|}
\newcommand{\VERB}{\Verb[commandchars=\\\{\}]}
\DefineVerbatimEnvironment{Highlighting}{Verbatim}{commandchars=\\\{\}}
% Add ',fontsize=\small' for more characters per line
\usepackage{framed}
\definecolor{shadecolor}{RGB}{248,248,248}
\newenvironment{Shaded}{\begin{snugshade}}{\end{snugshade}}
\newcommand{\AlertTok}[1]{\textcolor[rgb]{0.94,0.16,0.16}{#1}}
\newcommand{\AnnotationTok}[1]{\textcolor[rgb]{0.56,0.35,0.01}{\textbf{\textit{#1}}}}
\newcommand{\AttributeTok}[1]{\textcolor[rgb]{0.77,0.63,0.00}{#1}}
\newcommand{\BaseNTok}[1]{\textcolor[rgb]{0.00,0.00,0.81}{#1}}
\newcommand{\BuiltInTok}[1]{#1}
\newcommand{\CharTok}[1]{\textcolor[rgb]{0.31,0.60,0.02}{#1}}
\newcommand{\CommentTok}[1]{\textcolor[rgb]{0.56,0.35,0.01}{\textit{#1}}}
\newcommand{\CommentVarTok}[1]{\textcolor[rgb]{0.56,0.35,0.01}{\textbf{\textit{#1}}}}
\newcommand{\ConstantTok}[1]{\textcolor[rgb]{0.00,0.00,0.00}{#1}}
\newcommand{\ControlFlowTok}[1]{\textcolor[rgb]{0.13,0.29,0.53}{\textbf{#1}}}
\newcommand{\DataTypeTok}[1]{\textcolor[rgb]{0.13,0.29,0.53}{#1}}
\newcommand{\DecValTok}[1]{\textcolor[rgb]{0.00,0.00,0.81}{#1}}
\newcommand{\DocumentationTok}[1]{\textcolor[rgb]{0.56,0.35,0.01}{\textbf{\textit{#1}}}}
\newcommand{\ErrorTok}[1]{\textcolor[rgb]{0.64,0.00,0.00}{\textbf{#1}}}
\newcommand{\ExtensionTok}[1]{#1}
\newcommand{\FloatTok}[1]{\textcolor[rgb]{0.00,0.00,0.81}{#1}}
\newcommand{\FunctionTok}[1]{\textcolor[rgb]{0.00,0.00,0.00}{#1}}
\newcommand{\ImportTok}[1]{#1}
\newcommand{\InformationTok}[1]{\textcolor[rgb]{0.56,0.35,0.01}{\textbf{\textit{#1}}}}
\newcommand{\KeywordTok}[1]{\textcolor[rgb]{0.13,0.29,0.53}{\textbf{#1}}}
\newcommand{\NormalTok}[1]{#1}
\newcommand{\OperatorTok}[1]{\textcolor[rgb]{0.81,0.36,0.00}{\textbf{#1}}}
\newcommand{\OtherTok}[1]{\textcolor[rgb]{0.56,0.35,0.01}{#1}}
\newcommand{\PreprocessorTok}[1]{\textcolor[rgb]{0.56,0.35,0.01}{\textit{#1}}}
\newcommand{\RegionMarkerTok}[1]{#1}
\newcommand{\SpecialCharTok}[1]{\textcolor[rgb]{0.00,0.00,0.00}{#1}}
\newcommand{\SpecialStringTok}[1]{\textcolor[rgb]{0.31,0.60,0.02}{#1}}
\newcommand{\StringTok}[1]{\textcolor[rgb]{0.31,0.60,0.02}{#1}}
\newcommand{\VariableTok}[1]{\textcolor[rgb]{0.00,0.00,0.00}{#1}}
\newcommand{\VerbatimStringTok}[1]{\textcolor[rgb]{0.31,0.60,0.02}{#1}}
\newcommand{\WarningTok}[1]{\textcolor[rgb]{0.56,0.35,0.01}{\textbf{\textit{#1}}}}
\usepackage{graphicx,grffile}
\makeatletter
\def\maxwidth{\ifdim\Gin@nat@width>\linewidth\linewidth\else\Gin@nat@width\fi}
\def\maxheight{\ifdim\Gin@nat@height>\textheight\textheight\else\Gin@nat@height\fi}
\makeatother
% Scale images if necessary, so that they will not overflow the page
% margins by default, and it is still possible to overwrite the defaults
% using explicit options in \includegraphics[width, height, ...]{}
\setkeys{Gin}{width=\maxwidth,height=\maxheight,keepaspectratio}
\IfFileExists{parskip.sty}{%
\usepackage{parskip}
}{% else
\setlength{\parindent}{0pt}
\setlength{\parskip}{6pt plus 2pt minus 1pt}
}
\setlength{\emergencystretch}{3em}  % prevent overfull lines
\providecommand{\tightlist}{%
  \setlength{\itemsep}{0pt}\setlength{\parskip}{0pt}}
\setcounter{secnumdepth}{0}
% Redefines (sub)paragraphs to behave more like sections
\ifx\paragraph\undefined\else
\let\oldparagraph\paragraph
\renewcommand{\paragraph}[1]{\oldparagraph{#1}\mbox{}}
\fi
\ifx\subparagraph\undefined\else
\let\oldsubparagraph\subparagraph
\renewcommand{\subparagraph}[1]{\oldsubparagraph{#1}\mbox{}}
\fi

%%% Use protect on footnotes to avoid problems with footnotes in titles
\let\rmarkdownfootnote\footnote%
\def\footnote{\protect\rmarkdownfootnote}

%%% Change title format to be more compact
\usepackage{titling}

% Create subtitle command for use in maketitle
\providecommand{\subtitle}[1]{
  \posttitle{
    \begin{center}\large#1\end{center}
    }
}

\setlength{\droptitle}{-2em}

  \title{Rigdon Basu (1989) Time truncated data example}
    \pretitle{\vspace{\droptitle}\centering\huge}
  \posttitle{\par}
  \subtitle{replication in Stan}
  \author{Miao Cai \href{mailto:miao.cai@slu.edu}{\nolinkurl{miao.cai@slu.edu}}}
    \preauthor{\centering\large\emph}
  \postauthor{\par}
      \predate{\centering\large\emph}
  \postdate{\par}
    \date{1/29/2019}


\begin{document}
\maketitle

\hypertarget{time-truncated-data---115-kv-transmission-line-example}{%
\section{Time truncated data - 115 kV Transmission line
example}\label{time-truncated-data---115-kv-transmission-line-example}}

\hypertarget{this-data-is-presented-by-martz-1975}{%
\section{This data is presented by Martz
(1975)}\label{this-data-is-presented-by-martz-1975}}

\hypertarget{it-gives-the-failure-times-or-interrpution-times-of-the-115-kv-transmission-circuit-from-cunningham-generating-station-located-near-hobbs-new-mexico-to-eddy-county-interchange-located-near-artesia-new-mexico.-we-assume-that-data-collection-was-terminated-on-december-31-1971.}{%
\section{It gives the failure times, or interrpution times, of the 115
kV transmission circuit from Cunningham Generating Station, located near
Hobbs, New Mexico, to Eddy County Interchange, located near Artesia, New
Mexico. We assume that data collection was terminated on December 31,
1971.}\label{it-gives-the-failure-times-or-interrpution-times-of-the-115-kv-transmission-circuit-from-cunningham-generating-station-located-near-hobbs-new-mexico-to-eddy-county-interchange-located-near-artesia-new-mexico.-we-assume-that-data-collection-was-terminated-on-december-31-1971.}}

\hypertarget{data-collected-in-this-manner-with-testing-terminated-at-a-predetermined-time-are-called-time-truncated.-it-is-important-to-distinguish-between-these-approches-to-data-collection-because-statistical-inference-precedures-are-different-for-the-two-situations.}{%
\section{Data collected in this manner, with testing terminated at a
predetermined time, are called time truncated. It is important to
distinguish between these approches to data collection because
statistical inference precedures are different for the two
situations.}\label{data-collected-in-this-manner-with-testing-terminated-at-a-predetermined-time-are-called-time-truncated.-it-is-important-to-distinguish-between-these-approches-to-data-collection-because-statistical-inference-precedures-are-different-for-the-two-situations.}}

\begin{Shaded}
\begin{Highlighting}[]
\NormalTok{pacman}\OperatorTok{::}\KeywordTok{p_load}\NormalTok{(rstan, tidyverse)}


\NormalTok{t =}\StringTok{ }\KeywordTok{c}\NormalTok{(}\FloatTok{0.129}\NormalTok{, }\FloatTok{0.151}\NormalTok{, }\FloatTok{0.762}\NormalTok{, }\FloatTok{0.869}\NormalTok{, }\FloatTok{2.937}\NormalTok{, }\FloatTok{3.077}\NormalTok{, }\FloatTok{3.841}\NormalTok{, }\FloatTok{3.964}\NormalTok{, }\FloatTok{4.802}\NormalTok{, }\FloatTok{4.898}\NormalTok{, }\FloatTok{7.868}\NormalTok{, }\FloatTok{8.430}\NormalTok{)}
\NormalTok{y =}\StringTok{ }\KeywordTok{rep}\NormalTok{(}\FloatTok{0.5}\NormalTok{, }\KeywordTok{length}\NormalTok{(t))}
\NormalTok{dat =}\StringTok{ }\KeywordTok{data.frame}\NormalTok{(t, y)}
\NormalTok{trunc_time =}\StringTok{ }\FloatTok{8.463}

\KeywordTok{ggplot}\NormalTok{(dat, }\KeywordTok{aes}\NormalTok{(}\DataTypeTok{x =}\NormalTok{ t, }\DataTypeTok{y =}\NormalTok{ y)) }\OperatorTok{+}\StringTok{ }\KeywordTok{geom_point}\NormalTok{() }\OperatorTok{+}\StringTok{ }\KeywordTok{geom_line}\NormalTok{()}

\NormalTok{plpstan =}\StringTok{ '}
\StringTok{functions\{}
\StringTok{  real nhpp_log(vector t, real beta, real theta, real tau)\{}
\StringTok{    vector[num_elements(t)] loglik_part;}
\StringTok{    real loglikelihood;}
\StringTok{    for (i in 1:num_elements(t))\{}
\StringTok{      loglik_part[i] = log(beta) - beta*log(theta) + (beta - 1)*log(t[i]);}
\StringTok{    \}}
\StringTok{    loglikelihood = sum(loglik_part) - (tau/theta)^beta;}
\StringTok{    return loglikelihood;}
\StringTok{  \}}
\StringTok{\}}
\StringTok{data \{}
\StringTok{  int<lower=0> n; //total # of obs}
\StringTok{  real<lower=0> tau;//truncated time}
\StringTok{  vector<lower=0>[n] t; //failure time}
\StringTok{\}}
\StringTok{parameters\{}
\StringTok{  real<lower=0> beta;}
\StringTok{  real<lower=0> theta;}
\StringTok{\}}
\StringTok{model\{}
\StringTok{  t ~ nhpp(beta, theta, tau);}
\StringTok{//PRIORS}
\StringTok{beta ~ gamma(1, 1);}
\StringTok{theta ~ gamma(1, 1);}
\StringTok{\}}
\StringTok{'}

\NormalTok{datstan =}\StringTok{ }\KeywordTok{list}\NormalTok{(}\DataTypeTok{n =} \KeywordTok{length}\NormalTok{(t),}
               \DataTypeTok{tau =}\NormalTok{ trunc_time,}
               \DataTypeTok{t =}\NormalTok{ t)}

\NormalTok{fitplp <-}\StringTok{ }\KeywordTok{stan}\NormalTok{(}
  \DataTypeTok{model_code=}\NormalTok{plpstan, }\DataTypeTok{model_name=}\StringTok{"NHPP"}\NormalTok{, }\DataTypeTok{data=}\NormalTok{datstan, }
  \DataTypeTok{iter=}\DecValTok{5000}\NormalTok{,}\DataTypeTok{warmup =} \DecValTok{2000}\NormalTok{, }\DataTypeTok{chains=}\DecValTok{1}
\NormalTok{)}


\NormalTok{broom}\OperatorTok{::}\KeywordTok{tidy}\NormalTok{(fitplp)}
\end{Highlighting}
\end{Shaded}


\end{document}
